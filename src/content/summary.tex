\chapter{Összefoglalás}
\label{ch:sum}

Jelen dolgozat kiindulási problémáját az adta, hogy az IT biztonság területén használt SIEM, az IBM Security QRadar felhasználási lehetőségeit kibővítsük egy Idm, az IBM Security Identity Manager segítségével. A konkrét cél az volt, hogy a két termék közti integrációt egyedi megoldásokkal támogassuk, és az ISIM-ben elérhető, felhasználókkal kapcsolatos adatokat és folyamat információkat felkínáljuk a QRadar számára. Ezek segítségével elsősorban potenciálisan veszélyes tevékenységek detektálhatók az ISIM által kezelt menedzselt rendszereken, valamint az ISIM folyamatain belül.

Az integráció megvalósításához két módszert definiáltunk:
\begin{itemize}
	\item Releváns felhasználói adatok összegyűjtése és feltöltése a QRadar szabályrendszere számára.
	\item ISIM folyamat információiból események generálása, és ezek elküldése a QRadar számára.
\end{itemize}

A munkám előtt rendelkezésre állt már egy QRadar bővítmény, ami képes kezelni az ISIM-ből érkező audit események egy részhalmazát, ám ez nem volt felkészítve azokra a folyamat információkat tartalmazó események feldolgozására, amiket az általam készített megoldás kezel. Az általam megvalósított rendszer az IBM Tivoli Directory Integrator-t használja a betöltés, a generálás és a küldés végrehajtására, és a felhasznált adatokat a konkrét ISIM folyamatokból gyűjti össze. Ezekből egy syslog sort generálnak, amit a QRadar egy általam definiált esemény típussá dolgoz fel, ezáltal ezek további feldolgozása már a QRadarban elérhető módszerekkel elvégezhető.

A munkám előtt ISIM felhasználói adatainak integrációjára nem volt még kidolgozott megoldás, így ez egy teljesen új funkcionalitást biztosít. A felhasználói adatok feltöltéséhez a QRadar reference data adattárolóit használom, szintén a QRadar által biztosított REST API-on keresztül. Ez a feltöltési folyamat egy biztonságos, TLS titkosított csatornán keresztültörténik, aminél az applikációt egy API token hitelesíti a QRadar felé. A tényleges feltöltéshez készítettem egy Java wrapper osztályt, ami egy egyszerű, és könnyen használható interfészt biztosít a QRadar összes reference data típusának használatához, valamint könnyen kezelhetővé teszi az egyes plusz funkciók használatát is.

A feltöltő alkalmazás konkrét megvalósításához két architektúrát definiáltunk és valósítottunk meg, az egyik az IBM Tivoli Directory Integrator nevű adatintegrációs eszközt és framework-öt használja, a másik pedig egy Java webalkalmazás, ami IBM WebSphere alkalmazásszerveren fut. A TDI alapú megoldásnál én készítettem el az integrációhoz szükséges connectort a QRadar felé, valamint a minta integrációs feladatok egy részét.  Mivel a fejlesztés egy projekt keretében zajlott, és a TDI alapú és a WebSphere alapú megoldás párhuzamosan készült, ezért a Websphere alapú fejlesztésében csak a tervezési fázisban, a tesztelési fázisban, valamint az integrációs feladatok fejlesztésében vettem részt.

A fejlesztési folyamat eredményeként létrejött három megoldás, amik támogatják a leírt adatintegrációs feladatokat mind alkalmankénti futtatás, mint ütemezett, rendszeres futtatás esetén. Ezáltal a SIEM számára elérhetővé tettük az Idm rendszerben lévő felhasználókkal és jogosultságaikkal kapcsolatos adatokat, amely teljesebbé és pontosabbá teszi a SIEM által végzett biztonsági monitorozást.

