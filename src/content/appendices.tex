%----------------------------------------------------------------------------
\appendix
%----------------------------------------------------------------------------
\chapter*{\fuggelek}\addcontentsline{toc}{chapter}{\fuggelek}
\setcounter{chapter}{\appendixnumber}
%\setcounter{equation}{0} % a fofejezet-szamlalo az angol ABC 6. betuje (F) lesz
\numberwithin{equation}{section}
\numberwithin{figure}{section}
\numberwithin{lstlisting}{section}
\section{Feldolgozott attribútumok}
\begin{table}[!h]
	\centering
	\caption{Az ACTIVITY táblában található oszlopokból készített attribútumok}

	\begin{tabular}{lp{10cm}}
		Attribútum név & Attribútum funkciója \\
		\toprule
		A\_ACTIVITY\_INDEX & Activity számláló azonosítója, ha egy ciklusban szerepel \\
		A\_COMPLETED & Az activity befejezésének időpontja \\
		A\_DESCRIPTION & Maximum 300 karakteres, beszédes leírása az activity-nek \\
		A\_ID & Activity egyedi azonosítója \\
		A\_LASTMODIFIED & Az activity utolsó módosításának dátuma \\
		A\_LOCK\_COUNT & A függőben lévő feladatok az activity-n \\
		A\_LOOP\_COUNT & Iteráció specifikus érték. Az eltelt iterációk száma. \\
			A\_LOOP\_RUNCOUNT & Aszinkron ciklusos activity-k értéke. A hátralévő iterációk száma \\
		A\_NAME & Acvitiy beszédes neve. Segíti az activity azonosítását \\
		A\_PRIORITY & Activity prioritása. \\
		A\_PROCESS\_ID & Az adott activity-hez tartozó process egyedi azonosítója. Ezen keresztül köthető össze a process táblával \\
		A\_RESULT\_DETAIL & Végeredmény beszédes leírása \\
		A\_RESULT\_SUMMARY & Két karakteres leírása a végeredménynek. Ezzel a mezővel szűrhetünk az activity-k végeredményére. \\
		A\_RETRY\_COUNT & Próbálkozások száma az activity végrehajtására \\
		A\_SHORT\_DETAIL & Rövid, szöveges leírása az activity végeredményének \\
		A\_STARTED & Az activity indításának időpontja \\
		A\_STATE & Az activity aktuális állapota. Ezzel szűrhetünk pl a futó, a megállított, a befejezett, stb állapotokra. \\
		A\_SUBPROCESS\_ID & Az activity-hez kapcsolódó subprocess azonosítója. \\
		A\_SUBTYPE & Kézi activity-knél altípus, pl ezzel különböztethető meg a jóváhagyás/elutasítási, és az információ adási kérés. \\
		A\_TYPE & Az activity típusa, egy karakterként tárolva. Pl: kézi (M), alkalmazás (A), subprocess (S) \\	
	
\end{tabular}%

\end{table}

\begin{table}[b]
	 \centering
	 \caption{A PROCESS táblában található oszlopokból készített attribútumok}
	\begin{tabular}{lp{10cm}}
		Attribútum név & Attribútum funkciója \\
		\toprule
		P\_COMMENTS & Megjegyzések a process-hez \\
		P\_COMPLETED & A process befejezésének ideje. \\
	P\_DEFINITION\_ID & A process definíciójának azonosítója. \\
	P\_DESCRIPTION & Process szöveges leírása. \\
	P\_ID & Adott process azonosítója. \\
	P\_LASTMODIFIED & A process utolsó változásának időpontja. \\
	P\_NAME & A process neve. \\
	P\_NOTIFY & Megadja, hogy ki értesüljön a process befejeztekor. \\
	P\_PARENT\_ACTIVITY\_ID & Az adott processhez tartozó szülő activity azonosítója, ha van neki. \\
	P\_PARENT\_ID & A szülő process azonosítója, ha van neki. \\
	P\_PRIORITY & A process prioritása. \\
	P\_REQUESTEE & Annak a DN-je, aki számára a process végrehajtásra kerül. \\
	P\_REQUESTEE\_NAME & Annak a neve, aki számára a process végrehajtásra kerül. \\
	P\_REQUESTER & A process kérelmezőjének a DN-je. \\
	P\_REQUESTER\_NAME & A process kérelmezőjének a neve. \\
	P\_REQUESTER\_TYPE & A process kérelmezőjének a típusa. Pl végfelhasználó (U), ISIM System (P), Workflow (S). \\
	P\_RESULT\_DETAIL & Részletes információk a process eredményéről. \\
	P\_RESULT\_SUMMARY & A process végeredménye, két karakterként reprezentálva. Pl: jóváhagyva (AA), elutasítva (AR), sikertelen(SF). \\
	P\_ROOT\_PROCESS\_ID & A legmagasabb szinten álló szülő process azonosítója. \\
	P\_SCHEDULED & A process ütemezett indítási időpontja. \\
	P\_SHORT\_DETAIL & Rövid összefoglaló a process eredményéről. \\
	P\_STARTED & A process indításának ideje. \\
	P\_STATE & A process aktuális állapota. Pl: futó (R), kész (C), megállított (A) \\
	P\_SUBJECT & A process alanya. \\
	P\_SUBJECT\_ACCESS\_ID & A kérelmezett hozzáférés DN-je. (Ha a process hozzáférési kérelem volt) \\
	P\_SUBJECT\_ACCESS\_NAME & A kérelmezett hozzáférés neve. (Ha a process hozzáférési kérelem volt) \\
	P\_SUBJECT\_PROFILE & Az alany LDAP szintű típusa. \\
	P\_SUBJECT\_SERVICE & Az a service, amihez az alany tartozik. \\
	P\_SUBMITTED & A process létrehozásának időpontja. \\
	P\_TENANT & A kérelmezőhöz tartozó tartomány DN-je. \\
	P\_TYPE & 2 karakteres kódja a process típusának. Pl: új felhasználó (UA), jelszóváltoztatás (AP). \\
		
	\end{tabular}
\end{table}
\begin{table}
\small
\caption{Az összes feldolgozott property-t tartalmazó táblázat.} 

\begin{tabular}{lp{10cm}}
	Attribútum név & Attribútum funkciója \\
	\toprule
	PL\_ACTIVITY\_ID & A processlog eseményhez tartozó activity azonosítója. \\
	PL\_CREATED & A processlog esemény létrehozásának ideje. \\
	PL\_DATA\_ID & Az adat azonosítója adat változás esetén. \\
	PL\_EVENTTYPE & Az adott log esemény típus kódja. Pl: activity létrehozva (AC), activity állapot változott (AS) \\
	PL\_ID & Az adott process log esemény azonosítója. \\
	PL\_NEW\_DATA & Új adatok az adatváltozási esemény során, ha azok mérete nagyobb mint egy limit. \\
	PL\_NEW\_PARTICIPANT\_ID & Az új résztvevő azonosítója feladat delegáció esetén. \\
	PL\_NEW\_PARTICIPANT\_TYPE & Az új résztvevő típusa feladat delegáció esetén. \\
	PL\_NEW\_STATE & Új típus, egy típus változási esemény során. \\
	PL\_OLD\_PARTICIPANT\_ID & Feladat delegációs esemény esetén a régi résztvevő azonosítója. \\
	PL\_OLD\_PARTICIPANT\_TYPE & Feladat delegációs esemény esetén a régi résztvevő típusa. \\
	PL\_OLD\_STATE & Régi típus egy típus változási esemény során. \\
	PL\_PROCESS\_ID & A process azonosítója, amihez a process log esemény tartozik. \\
	PL\_REQUESTOR & Igénylő neve felhasználókkal kapcsolatos eseményeknél \\
	PL\_REQUESTOR\_DN & Igénylő DN-je felhasználókkal kapcsolatos eseményeknél \\
	PL\_REQUESTOR\_TYPE & Igénylő típusa felhasználókkal kapcsolatos eseményeknél \\
	PL\_SMALL\_NEW\_DATA & Új adatok az adatváltozási esemény során, ha azok mérete kisebb mind egy limit. \\
\end{tabular}
\end{table}


%	\label{tab:allprops}%
%\end{table}%

%----------------------------------------------------------------------------
%\clearpage\section{Válasz az ,,Élet, a világmindenség, meg minden'' kérdésére}
%----------------------------------------------------------------------------
%A Pitagorasz-tételből levezetve
%\begin{align}
%c^2=a^2+b^2=42.
%\end{align}
%A Faraday-indukciós törvényből levezetve
%\begin{align}
%\rot E=-\frac{dB}{dt}\hspace{1cm}\longrightarrow \hspace{1cm}
%U_i=\oint\limits_\mathbf{L}{\mathbf{E}\mathbf{dl}}=-\frac{d}{dt}\int\limits_A{\mathbf{B}\mathbf{da}}=42.
%\end{align}
