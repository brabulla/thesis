\pagenumbering{roman}
\setcounter{page}{1}

\selecthungarian

%----------------------------------------------------------------------------
% Abstract in Hungarian
%----------------------------------------------------------------------------
\chapter*{Kivonat}\addcontentsline{toc}{chapter}{Kivonat}

Az informatika fejlődésével és terjedésével egyre fontosabb terület lesz az informatikai biztonságtechnika ami rengeteg szerteágazó feladatot foglal magába. Jelen dolgozat ezen belül kettő területre, a Biztonsági incidensek és események kezelésére, valamint a Felhasználó és hozzáférés menedzsmentre fókuszál. Mindkét területre léteznek már ipari sztenderd megoldások, mint például az IBM QRadar, és Identity Manager terméke, azonban a két terület összekapcsolására, a felhasználói információk felhasználására biztonsági események kiértékelésénél, nem volt még megoldás. 

A dolgozat célja egy ilyen integráció elkészítése és bemutatása a fent említett két termék használatával. A cél a felhasználói, és a hozzájuk kötődő folyamat információk összegyűjtése, transzformálása, majd prezentálása a QRadar számára az események kiértékeléséhez. A dolgozat az információk elérhetővé tételére két megoldást mutat be: egy JAVA EE alapú webalkalmazást, és egy IBM adatintegrációs eszköz alapú megoldást. A két rendszer további összekapcsolására az említett mellett egy saját eseményforrást is készítettem, ami olyan információkat használ fel a felhasználói folyamatokról, amik eddig nem jutottak el a QRadar-hoz.


\vfill
\selectenglish


%----------------------------------------------------------------------------
% Abstract in English
%----------------------------------------------------------------------------
\chapter*{Abstract}\addcontentsline{toc}{chapter}{Abstract}

This document is a \LaTeX-based skeleton for BSc/MSc~theses of students at the Electrical Engineering and Informatics Faculty, Budapest University of Technology and Economics. The usage of this skeleton is optional. It has been tested with the \emph{TeXLive} \TeX~implementation, and it requires the PDF-\LaTeX~compiler.


\vfill
\selectthesislanguage

\newcounter{romanPage}
\setcounter{romanPage}{\value{page}}
\stepcounter{romanPage}