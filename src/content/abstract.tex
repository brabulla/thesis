\pagenumbering{roman}
\setcounter{page}{1}

\selecthungarian

%----------------------------------------------------------------------------
% Abstract in Hungarian
%----------------------------------------------------------------------------
\chapter*{Kivonat}\addcontentsline{toc}{chapter}{Kivonat}

Az informatika fejlődésével és terjedésével egyre fontosabb terület lesz az informatikai biztonságtechnika ami rengeteg szerteágazó feladatot foglal magába. Jelen dolgozat ezen belül kettő területre, a biztonsági incidensek és események kezelésére, valamint a felhasználó és hozzáférés menedzsmentre fókuszál. Mindkét területre léteznek már ipari sztenderd megoldások, mint például az IBM QRadar, és Identity Manager terméke, azonban a két terület összekapcsolására nem volt még megoldás. 

A dolgozat célja egy ilyen integráció elkészítése és bemutatása a fent említett két termék használatával. A cél a felhasználói, és a hozzájuk kötődő folyamat információk összegyűjtése, transzformálása, majd prezentálása a QRadar számára a szabályrendszerében való felhasználásra. Ezek segítségével az egyes események új kontextusban, új információk birtokában értékelhetők ki, ami eddig nem kezelt biztonsági incidensek észlelését teszi lehetővé.


A dolgozat az információk elérhetővé tételére két megoldást mutat be: egy JAVA EE alapú webalkalmazást, és egy IBM adatintegrációs eszköz alapú megoldást. A két rendszer közti további integrációra az említett mellett egy saját megoldást is készítettem, ami olyan, a felhasználói folyamatokkal kapcsolatos információkat továbbít a QRadar számára,  amik eddig kihasználatlanok voltak.


\vfill
\selectenglish


%----------------------------------------------------------------------------
% Abstract in English
%----------------------------------------------------------------------------
\chapter*{Abstract}\addcontentsline{toc}{chapter}{Abstract}

With the continous advancement of computers and technology, IT Security is becoming an increasingly important field of study, which envelops a multitude of tasks. This thesis focuses on two fields of IT security: security information and event management, and identity and access management. Both of these have had industry standard solutions for years, like IBM QRadar, and IBM Security Identity Manager, but there were no previous solutions for connecting the two.

The goal of this thesis project is to implement and present a solution for this problem, using the aforementioned two products. The end goal is the collection, transformation, and presentation of user identity and access information, including the processes managing these, for QRadar. With this data, the security events handled by QRadar can be evaluatied in a new context, detecting previously unhandled incidents.

This thesis presents two solutions for this integration: a Java EE based web application, and a custom solution, using an IBM data integration framework. For further integration between the two products, I've also developed a different integration solution, which generates security events for QRadar, from previously unutilized information about processes managing user data and access information.


\vfill
\selectthesislanguage

\newcounter{romanPage}
\setcounter{romanPage}{\value{page}}
\stepcounter{romanPage}