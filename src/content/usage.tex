\chapter{Megoldások telepítése, használata}
A telepítés és a használat leírásának során feltételezem, hogy már fel van telepítve, és rendelkezésre áll a felhasználni kívánt QRadar, az ISIM, a TDI, valamint a megfelelő WebSphere. A fejlesztés és a tesztelés során felhasznált applikációk verziói a következők:

\begin{itemize}
	\item QRadar 7.2.8
	\item ISIM 6.0
	\item TDI 7.1.1
	\item WebSphere liberty 17.0.0.2
\end{itemize}

\section{TDI alapú integrációs modul}
A megoldás általam készített része egy wrapper, a hozzá tartozó segédosztályokkal, valamint egy TDI connector implementáció, ami a wrapperre épül. A wrapper önmagában is felhasználható más projektekben. A connector egy .jar formájában használható fel, amit a megfelelő TDI installáció \$INSTALL\_DIR\$/jars/connectors mappájába bemásolva használhatunk fel. 
\subsection{Dependenciák}
A wrapper a HTTP hívások bonyolítására az Apache Wink \footnote{\href{https://wink.apache.org/}{Apache Wink hivatalos oldala: https://wink.apache.org/}} framework-öt használja, így ezen a library-n kívül szükség van ennek a dependenciáira is, mint például a J2EE bővítményekre.

A wrapper ezen kívül logolásra az SLF4J\footnote{\href{https://www.slf4j.org/}{SLF4J hivatalos oldala: https://www.slf4j.org/}} könyvtárat használja.

Mivel a QRadar irányú kapcsolat SSL titkosított és ehhez 2048 bites kulcsot használ, valamint DHE kulcscserét, ezért bizonyos java verziók esetén hibák léphetnek fel az SSL Handshake folyamán
\footnote{\href{https://stackoverflow.com/questions/6851461/java-why-does-ssl-handshake-give-could-not-generate-dh-keypair-exception}{Megoldások 1.6-os Java esetén: https://stackoverflow.com/questions/6851461/java-why-does-ssl-handshake-give-could-not-generate-dh-keypair-exception}} \footnote{\href{https://developer.ibm.com/answers/questions/209245/ssl-exception-error-in-wesphere-application-server.html}{Megoldás 1.6-os Java és IBM WebSphere használata esetén: https://developer.ibm.com/answers/questions/209245/ssl-exception-error-in-wesphere-application-server.html}}.
Ajánlott a Java 1.7-es verzióját használni, amiben ezek már javítva vannak.
%milyen depek vannak -> JaxRS, java2ee, slf4j, apache wink.
\subsection{Telepítés}

\subsubsection{Fordítás}
Ha csak forráskód áll rendelkezésre, akkor szükséges a projekt lefordítása, és egy .jar fájlba csomagolása. A fordításhoz szükségesek a fent felsorolt dependenciák elérhetővé tétele, valamint kettő, a TDI által biztosított .jar fájl: \textit{miserver.jar} és \textit{miconfig.jar}. Ezek megtalálhatók a \textit{\$TDI\_INSTALL\$/jars/common} mappában. 
Az elkészült jar fájl tartalmazza a fordított .class fájlokat a megfelelő mappa struktúrában, valamint a gyökérkönyvtárban a \ref{subsec:connimpl}. fejezetben leírtaknak megfelelően elkészített tdi.xml-t.
\subsubsection{TDI oldali konfiguráció}
Ahhoz, hogy a TDI használni tudja a connector-t, az elkészített .jar fájlt be kell másolni a \textit{\$TDI\_INSTALL\$/jars/connectors} mappába, valamint a szükséges dependenciákat a \textit{\$TDI\_INSTALL\$/jars/3rdparty/others} mappába.

\subsection{SSL titkosítás beállítása}
Mivel a QRadar SSL titkosítást használ, amihez egy self-signed certificate-el rendelkezik, ezt a certificate-et hozzá kell adni a TDI által használt SSL keystore-okhoz. Erre a feladatra ajánlott a TDI-al együtt érkező Java installáció által biztosított Ikeyman\footnote{Elérési út: \textit{\$TDI\_install\_dir\$/jvm/jre/bin}} grafikus alkalmazás használata. A két keystore, amihez a kulcsokat hozzá kell adni a \textit{\$TDI\_install\_dir\$/testserver.jks} valamint a \textit{\$TDI\_install\_dir\$/serverapi/testadmin.jks} \footnote{\href{https://www.ibm.com/support/knowledgecenter/en/SSCQG\_7.1.1/com.ibm.IBMDI.do\_7.1.1/adminguide36.htm}{A procedúráról bővebb leírás található az alábbi linken: https://www.ibm.com/support/ knowledgecenter/en/SSCQG\_7.1.1/com.ibm.IBMDI.do\_7.1.1/adminguide36.htm}}.

\subsubsection{QRadar oldali beállítás}
A TDI connector - QRadar irányú authentikáció biztosításához szükséges egy QRadar API kulcs. Ezt a QRadar webes admin felületén, az \textit{Admin -> User Management -> Authorized Services} menüpont alatt található. Itt egy új rekord felvétele és felkonfigurálása után, az Authentication Token mezőben található token-t használva a connector felkonfigurálásához létrehozható a kapcsolat a REST API-val. 

% mit hova másoljunk, cert hozzáadása a megfelelő storehoz, stb