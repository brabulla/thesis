\chapter{Megoldások telepítése, használata}
A telepítés és a használat leírásának során feltételezem, hogy már fel van telepítve, és rendelkezésre áll a felhasználni kívánt QRadar, az ISIM, a TDI, valamint a megfelelő WebSphere. A fejlesztés és a tesztelés során felhasznált applikációk verziói a következők:

\begin{itemize}
	\item QRadar 7.2.8
	\item ISIM 6.0
	\item TDI 7.1.1
	\item WebSphere liberty 17.0.0.2
\end{itemize}

\section{TDI alapú integrációs modul}
A megoldás általam készített része egy wrapper, a hozzá tartozó segédosztályokkal, valamint egy TDI connector implementáció, ami a wrapperre épül. A wrapper önmagában is felhasználható más projektekben. A connector egy .jar formájában használható fel, amit a megfelelő TDI installáció \$INSTALL\_DIR\$/jars/connectors mappájába bemásolva használhatunk fel. 
\subsection{Dependenciák}
A wrapper a HTTP hívások bonyolítására az Apache Wink \footnote{\href{https://wink.apache.org/}{Apache Wink hivatalos oldala}} framework-öt használja, így ezen a library-n kívül szükség van ennek a dependenciáira is, mint például a J2EE bővítményekre.

A wrapper ezen kívül logolásra az SLF4J\footnote{\href{https://www.slf4j.org/}{SLF4J hivatalos oldala}} könyvtárat használja.

Mivel a QRadar irányú kapcsolat SSL titkosított és ehhez 2048 bites kulcsot használ, valamint DHE kulcscserét, ezért bizonyos java verziók esetén hibák léphetnek fel az SSL Handshake folyamán
\footnote{\href{https://stackoverflow.com/questions/6851461/java-why-does-ssl-handshake-give-could-not-generate-dh-keypair-exception}{Megoldások 1.6-os Java esetén}} \footnote{\href{https://developer.ibm.com/answers/questions/209245/ssl-exception-error-in-wesphere-application-server.html} Megoldás 1.6-os Java és IBM WebSphere használata esetén}} 
%milyen depek vannak -> JaxRS, java2ee, slf4j, apache wink.
\subsection{Telepítés}
% mit hova másoljunk, cert hozzáadása a megfelelő storehoz, stb